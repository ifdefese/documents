\documentclass{report}
\begin{document}

\begin{section}{Source-code Management}

As this is going to be a largely software-based project, there must be some
underlying mechanism for collaboration on developing the source code. The most
common method of collaboration is through the use of version control software.
In addition to this, a code review system will be used to ensure all code
submissions are checked for quality.

\begin{subsection}{Version Control}

The version control system employed will be Git. Git is widely used in
open-source, and was originally designed for the development of the Linux
kernel. It allows developers to have their own local copy of the source code,
complete with the benefits of version control. Developers edit the local copy
of the code, and then "commit" the changes, which creates a "commit record"
detailing what changed and officially saving it into the repository. At a later
date, users can then "push" these changes to another repository, or pull
changes from repositories. This will cause missing commit records to be
synchronised between the repositories and will update the files which have
changes. \\

The general workflow for version contro will be as follows:\\

\begin{itemize}
\item A central repository will exist on a server, which will hold the latest state of all source code
\item Developers will "clone" this repository and commit changes to it
\item Developers can then "re-base" changes to merge any commits that would be
better represented as a single change, or remove unnecessary changes from
commits.
\item Developers use "push" to send their changes to the central repository
\item Developers use "pull" to get changes from the central repository
\end{itemize}

\end{subsection}

\begin{subsection}{Code Review}

The Code Review system employed will be Gerrit. Gerrit integrates well with
Git, and enforces all changes being pushed to be checked first by other team
members before the changes are applied to the central repository. Reviewers can
make comments on the commits in their entirety, or on specific lines in
specific files which have changes. It also allows commits to be modified and
updated before being applied, if reviewers decide this is necessary. \\

The general code review workflow will be as follows:\\

\begin{itemize}
\item A central Gerrit server will host all Git repositories
\item Developers "push" their changes to a review branch
\item Gerrit creates code review tickets for each commit
\item Developer selects some reviewers
\item Reviewers make comments on code
\item Developer re-factors code if necessary, and pushes again
\item Reviewers approve changes
\item Code is merged into main repository
\end{itemize}

\end{subsection}

\end{section}
\end{document}
